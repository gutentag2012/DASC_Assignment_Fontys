% Options for packages loaded elsewhere
\PassOptionsToPackage{unicode}{hyperref}
\PassOptionsToPackage{hyphens}{url}
%
\documentclass[
]{article}
\usepackage{lmodern}
\usepackage{amssymb,amsmath}
\usepackage{ifxetex,ifluatex}
\ifnum 0\ifxetex 1\fi\ifluatex 1\fi=0 % if pdftex
  \usepackage[T1]{fontenc}
  \usepackage[utf8]{inputenc}
  \usepackage{textcomp} % provide euro and other symbols
\else % if luatex or xetex
  \usepackage{unicode-math}
  \defaultfontfeatures{Scale=MatchLowercase}
  \defaultfontfeatures[\rmfamily]{Ligatures=TeX,Scale=1}
\fi
% Use upquote if available, for straight quotes in verbatim environments
\IfFileExists{upquote.sty}{\usepackage{upquote}}{}
\IfFileExists{microtype.sty}{% use microtype if available
  \usepackage[]{microtype}
  \UseMicrotypeSet[protrusion]{basicmath} % disable protrusion for tt fonts
}{}
\makeatletter
\@ifundefined{KOMAClassName}{% if non-KOMA class
  \IfFileExists{parskip.sty}{%
    \usepackage{parskip}
  }{% else
    \setlength{\parindent}{0pt}
    \setlength{\parskip}{6pt plus 2pt minus 1pt}}
}{% if KOMA class
  \KOMAoptions{parskip=half}}
\makeatother
\usepackage{xcolor}
\IfFileExists{xurl.sty}{\usepackage{xurl}}{} % add URL line breaks if available
\IfFileExists{bookmark.sty}{\usepackage{bookmark}}{\usepackage{hyperref}}
\hypersetup{
  pdftitle={BKA\_Kriminalstatistik\_2019\_data\_analysis},
  pdfauthor={Peter von Bodelschwingh, Joshua Gawenda},
  hidelinks,
  pdfcreator={LaTeX via pandoc}}
\urlstyle{same} % disable monospaced font for URLs
\usepackage[margin=1in]{geometry}
\usepackage{color}
\usepackage{fancyvrb}
\newcommand{\VerbBar}{|}
\newcommand{\VERB}{\Verb[commandchars=\\\{\}]}
\DefineVerbatimEnvironment{Highlighting}{Verbatim}{commandchars=\\\{\}}
% Add ',fontsize=\small' for more characters per line
\usepackage{framed}
\definecolor{shadecolor}{RGB}{248,248,248}
\newenvironment{Shaded}{\begin{snugshade}}{\end{snugshade}}
\newcommand{\AlertTok}[1]{\textcolor[rgb]{0.94,0.16,0.16}{#1}}
\newcommand{\AnnotationTok}[1]{\textcolor[rgb]{0.56,0.35,0.01}{\textbf{\textit{#1}}}}
\newcommand{\AttributeTok}[1]{\textcolor[rgb]{0.77,0.63,0.00}{#1}}
\newcommand{\BaseNTok}[1]{\textcolor[rgb]{0.00,0.00,0.81}{#1}}
\newcommand{\BuiltInTok}[1]{#1}
\newcommand{\CharTok}[1]{\textcolor[rgb]{0.31,0.60,0.02}{#1}}
\newcommand{\CommentTok}[1]{\textcolor[rgb]{0.56,0.35,0.01}{\textit{#1}}}
\newcommand{\CommentVarTok}[1]{\textcolor[rgb]{0.56,0.35,0.01}{\textbf{\textit{#1}}}}
\newcommand{\ConstantTok}[1]{\textcolor[rgb]{0.00,0.00,0.00}{#1}}
\newcommand{\ControlFlowTok}[1]{\textcolor[rgb]{0.13,0.29,0.53}{\textbf{#1}}}
\newcommand{\DataTypeTok}[1]{\textcolor[rgb]{0.13,0.29,0.53}{#1}}
\newcommand{\DecValTok}[1]{\textcolor[rgb]{0.00,0.00,0.81}{#1}}
\newcommand{\DocumentationTok}[1]{\textcolor[rgb]{0.56,0.35,0.01}{\textbf{\textit{#1}}}}
\newcommand{\ErrorTok}[1]{\textcolor[rgb]{0.64,0.00,0.00}{\textbf{#1}}}
\newcommand{\ExtensionTok}[1]{#1}
\newcommand{\FloatTok}[1]{\textcolor[rgb]{0.00,0.00,0.81}{#1}}
\newcommand{\FunctionTok}[1]{\textcolor[rgb]{0.00,0.00,0.00}{#1}}
\newcommand{\ImportTok}[1]{#1}
\newcommand{\InformationTok}[1]{\textcolor[rgb]{0.56,0.35,0.01}{\textbf{\textit{#1}}}}
\newcommand{\KeywordTok}[1]{\textcolor[rgb]{0.13,0.29,0.53}{\textbf{#1}}}
\newcommand{\NormalTok}[1]{#1}
\newcommand{\OperatorTok}[1]{\textcolor[rgb]{0.81,0.36,0.00}{\textbf{#1}}}
\newcommand{\OtherTok}[1]{\textcolor[rgb]{0.56,0.35,0.01}{#1}}
\newcommand{\PreprocessorTok}[1]{\textcolor[rgb]{0.56,0.35,0.01}{\textit{#1}}}
\newcommand{\RegionMarkerTok}[1]{#1}
\newcommand{\SpecialCharTok}[1]{\textcolor[rgb]{0.00,0.00,0.00}{#1}}
\newcommand{\SpecialStringTok}[1]{\textcolor[rgb]{0.31,0.60,0.02}{#1}}
\newcommand{\StringTok}[1]{\textcolor[rgb]{0.31,0.60,0.02}{#1}}
\newcommand{\VariableTok}[1]{\textcolor[rgb]{0.00,0.00,0.00}{#1}}
\newcommand{\VerbatimStringTok}[1]{\textcolor[rgb]{0.31,0.60,0.02}{#1}}
\newcommand{\WarningTok}[1]{\textcolor[rgb]{0.56,0.35,0.01}{\textbf{\textit{#1}}}}
\usepackage{graphicx,grffile}
\makeatletter
\def\maxwidth{\ifdim\Gin@nat@width>\linewidth\linewidth\else\Gin@nat@width\fi}
\def\maxheight{\ifdim\Gin@nat@height>\textheight\textheight\else\Gin@nat@height\fi}
\makeatother
% Scale images if necessary, so that they will not overflow the page
% margins by default, and it is still possible to overwrite the defaults
% using explicit options in \includegraphics[width, height, ...]{}
\setkeys{Gin}{width=\maxwidth,height=\maxheight,keepaspectratio}
% Set default figure placement to htbp
\makeatletter
\def\fps@figure{htbp}
\makeatother
\setlength{\emergencystretch}{3em} % prevent overfull lines
\providecommand{\tightlist}{%
  \setlength{\itemsep}{0pt}\setlength{\parskip}{0pt}}
\setcounter{secnumdepth}{-\maxdimen} % remove section numbering

\title{BKA\_Kriminalstatistik\_2019\_data\_analysis}
\author{Peter von Bodelschwingh, Joshua Gawenda}
\date{12 12 2020}

\begin{document}
\maketitle

\hypertarget{german-crime-statistics-2019---a-data-analysis}{%
\section{German crime statistics 2019 - A data
analysis}\label{german-crime-statistics-2019---a-data-analysis}}

\hypertarget{introduction}{%
\subsection{Introduction}\label{introduction}}

In this data analysis we are going to take a look at the German crime
statistics from the year 2019, the special version we are looking at
also shows the crime distribution about different sized cities. The data
is divided for all 16 German provinces and into a general section for
the whole country. + There are also multiple files that help to
determine what can be deducted from a data analysis of this set and what
is out of scope. + The data set is taken from the ``Bundeskriminalamt''
- the leading institution of crime detection in Germany. + The original
data set can be found here:
\href{https://www.bka.de/DE/AktuelleInformationen/StatistikenLagebilder/PolizeilicheKriminalstatistik/PKS2019/PKSTabellen/LandFalltabellen/landFalltabellen.html}{BKA
Dataset} + This data set only shows absolute numbers for all provinces
in Germany, so to make the statistics comparable over all the different
provinces we also included another data set, that shows the number of
inhabitants for every province and the whole country. + This data set
can be found here:
\href{https://www.destatis.de/DE/Themen/Laender-Regionen/Regionales/Gemeindeverzeichnis/Administrativ/02-bundeslaender.html}{German
Inhabitants} + Later you will see that we imported on more data set.
This one is generated out of a PDF from the documentation for the first
data set and therefore is not available in the format we have online. It
holds information about the so called sum keys of the main data set and
is later used to decouple the main data set.

\hypertarget{sum-keys}{%
\subsubsection{Sum Keys}\label{sum-keys}}

To understand a big part of the analysis you have to understand what sum
keys are, this is what we are going to describe in this chapter. + A key
in the data set we use represents one type of crime, every crime has its
own distinct key. Some crimes are just a composite of multiple other
crimes, their key is then called a sum key, so every crime that is not a
sum, has no children but only a parent crime. + A key is made out of six
digits, where every digit is read as a single one from left to right. In
general every key that starts with the same digits is part of the same
sum key, the sum key is then identified with the same starting digits
followed by zeros. However this is not true in every case, some keys
that end with zero have no children and therefore are no sum keys, the
sum key for those keys is in most cases represented by a "*"-sign. + One
general section of crimes is therefore represented by only the first
digit. There are also two whole sections of keys that are only made out
of sum keys and are used by the police to study different fields of
crimes.

\hypertarget{import}{%
\subsection{Import}\label{import}}

The first step to every data analysis is to import the data into your
project. The data we chose is available as an excel file so we use the
package ``readxl'' to import the sets from excel. For the import to work
properly we had to remove some unnecessary header lines in the main data
set, we did that by specifying a ``range'' in the read\_excel command. +
The ``SumKeys'' data set was created by us, therefore every optimization
for importing where made while creating the set. + Finally we also
import the data set for the inhabitants, since this data set is not part
of the main analysis, we decided to also edit it in excel to optimize it
for the import and remove data that we do not need for the analysis.

\begin{Shaded}
\begin{Highlighting}[]
\NormalTok{originalSet <-}\StringTok{ }\KeywordTok{read_excel}\NormalTok{(}\StringTok{"DataSets/PKS_crime_statistics.xlsx"}\NormalTok{, }\DataTypeTok{sheet =} \StringTok{"Data"}\NormalTok{,}\DataTypeTok{range =} \StringTok{"A8:U18777"}\NormalTok{) }
\NormalTok{sumKeysImport <-}\StringTok{ }\KeywordTok{read_excel}\NormalTok{(}\StringTok{"DataSets/PKS_crime_statistics.xlsx"}\NormalTok{, }\DataTypeTok{sheet =} \StringTok{"Sumkeys"}\NormalTok{)}
\NormalTok{germanInhabitants <-}\StringTok{ }\KeywordTok{read_excel}\NormalTok{(}\StringTok{"DataSets/02-bundeslaender.xlsx"}\NormalTok{, }\DataTypeTok{sheet =} \StringTok{"inhabitants"}\NormalTok{)}
\end{Highlighting}
\end{Shaded}

\hypertarget{renaming}{%
\subsubsection{Renaming}\label{renaming}}

After the import, the names of the original data set got lost and are
not descriptive anymore. To assure that the usage later on is more
understandable we renamed the data set.

\begin{Shaded}
\begin{Highlighting}[]
\NormalTok{originalSet <-}\StringTok{ }\NormalTok{originalSet }\OperatorTok\StringTok{ }
\StringTok{                }\KeywordTok{rename}\NormalTok{(}
                  \DataTypeTok{key =} \StringTok{'1'}\NormalTok{,}
                  \DataTypeTok{crime =} \StringTok{'2'}\NormalTok{,}
                  \DataTypeTok{province =} \StringTok{'3'}\NormalTok{,}
                  \DataTypeTok{detected_cases =} \StringTok{'4'}\NormalTok{,}
                  \DataTypeTok{detected_cases_percent =} \StringTok{'5'}\NormalTok{,}
                  \DataTypeTok{not_executed_cases =} \StringTok{'6'}\NormalTok{,}
                  \DataTypeTok{not_executed_cases_percent =} \StringTok{'7'}\NormalTok{,}
                  \DataTypeTok{distribution_under_20K =} \StringTok{'8'}\NormalTok{,}
                  \DataTypeTok{distribution_20K_to_100K =} \StringTok{'9'}\NormalTok{,}
                  \DataTypeTok{distribution_100K_to_500K =} \StringTok{'10'}\NormalTok{,}
                  \DataTypeTok{distribution_over_500K =} \StringTok{'11'}\NormalTok{,}
                  \DataTypeTok{distribution_unknown =} \StringTok{'12'}\NormalTok{,}
                  \DataTypeTok{gun_threatened =} \StringTok{'13'}\NormalTok{,}
                  \DataTypeTok{gun_shot =} \StringTok{'14'}\NormalTok{,}
                  \DataTypeTok{solved_cases =} \StringTok{'15'}\NormalTok{,}
                  \DataTypeTok{solved_cases_percent =} \StringTok{'16'}\NormalTok{,}
                  \DataTypeTok{suspects =} \StringTok{'17'}\NormalTok{,}
                  \DataTypeTok{male =} \StringTok{'18'}\NormalTok{,}
                  \DataTypeTok{female =} \StringTok{'19'}\NormalTok{,}
                  \DataTypeTok{non_german =}\StringTok{'20'}\NormalTok{,}
                  \DataTypeTok{non_german_percent =} \StringTok{'21'}
\NormalTok{                  )}
\end{Highlighting}
\end{Shaded}

As a last step of importing, we can merge the inhabitants data frame
into the original data frame for the analysis. This would allow to use
the number of inhabitants easily in any calculation from here on.

\begin{Shaded}
\begin{Highlighting}[]
\NormalTok{originalSet <-}\StringTok{ }\KeywordTok{merge}\NormalTok{(originalSet, germanInhabitants, }\DataTypeTok{by =} \StringTok{"province"}\NormalTok{) }\OperatorTok\StringTok{ }\KeywordTok{arrange}\NormalTok{(key)}
\KeywordTok{head}\NormalTok{(originalSet)}
\end{Highlighting}
\end{Shaded}

\begin{verbatim}
##                     province    key                crime detected_cases
## 1          Baden-Württemberg ------ Straftaten insgesamt         573813
## 2                     Bayern ------ Straftaten insgesamt         603464
## 3                     Berlin ------ Straftaten insgesamt         513426
## 4                Brandenburg ------ Straftaten insgesamt         171828
## 5                     Bremen ------ Straftaten insgesamt          78228
## 6 Bundesrepublik Deutschland ------ Straftaten insgesamt        5436401
##   detected_cases_percent not_executed_cases not_executed_cases_percent
## 1                    100              48562                        8.5
## 2                    100              45581                        7.6
## 3                    100              41505                        8.1
## 4                    100              11278                        6.6
## 5                    100               7288                        9.3
## 6                    100             416594                        7.7
##   distribution_under_20K distribution_20K_to_100K distribution_100K_to_500K
## 1                 175726                   207084                    132301
## 2                 235721                   141835                     64933
## 3                      0                        0                         0
## 4                  81968                    64415                     24717
## 5                      0                        0                     12020
## 6                1243451                  1451426                   1061629
##   distribution_over_500K distribution_unknown gun_threatened gun_shot
## 1                  54347                 4355            274      325
## 2                 129814                31161            354      345
## 3                 513426                    0            342      316
## 4                      0                  728             88      124
## 5                  65869                  339             67       43
## 6                1569157               110738           4512     4639
##   solved_cases solved_cases_percent suspects    male female non_german
## 1       348664                 60.8   238737  182574  56163      96219
## 2       404145                 67.0   289856  221025  68831     121928
## 3       229532                 44.7   136704  100677  36027      61841
## 4        96690                 56.3    65374   49299  16075      16303
## 5        38118                 48.7    24101   18108   5993       9599
## 6      3124161                 57.5  2019211 1514667 504544     699261
##   non_german_percent inhabitants
## 1               40.3    11100394
## 2               42.1    13124737
## 3               45.2     3669491
## 4               24.9     2521893
## 5               39.8      681202
## 6               34.6    83166711
\end{verbatim}

\hypertarget{tidy}{%
\subsection{Tidy}\label{tidy}}

This next step includes checking for missing values and in our case
splitting the data frames up, so that the provinces are decoupled from
the values for whole Germany. We also need to remove all sum keys from
the actual capture keys.

\hypertarget{check-non-na}{%
\subsubsection{Check non NA}\label{check-non-na}}

If the data set contains empty values they are marked as NA and have to
be handled somehow. However, this data set has no such values as shown
in the next code chunk. Here we filter for NA values in any column and
check if the row count is 0 in the end.

\begin{Shaded}
\begin{Highlighting}[]
\KeywordTok{nrow}\NormalTok{(}\KeywordTok{filter_at}\NormalTok{(originalSet, }\DecValTok{1}\OperatorTok{:}\DecValTok{21}\NormalTok{, }\KeywordTok{any_vars}\NormalTok{(}\KeywordTok{is.na}\NormalTok{(.)))) }\OperatorTok{==}\StringTok{ }\DecValTok{0}
\end{Highlighting}
\end{Shaded}

\begin{verbatim}
## [1] TRUE
\end{verbatim}

\hypertarget{decouple}{%
\subsubsection{Decouple}\label{decouple}}

Like describe in the ``Sum Key'' section, the original data set is a mix
of sum keys and capture keys. In order to not count twice in some
operations we remove the sum keys from the capture keys, so the main
analysis can focus on the capture keys. + The sum keys set that we
imported has two columns, one for the key and one with the value ``Y''
for every capture key and a ``N'' for every sum key. The two key
sections that are only made out of sum keys are not reported in the set,
so can be handled as ``NA''.

\begin{Shaded}
\begin{Highlighting}[]
\NormalTok{captureKeys <-}\StringTok{ }\NormalTok{sumKeysImport }\OperatorTok\StringTok{ }\KeywordTok{filter}\NormalTok{(}\KeywordTok{grepl}\NormalTok{(}\StringTok{"Y"}\NormalTok{,is_sum_key))}
\NormalTok{sumKeys <-}\StringTok{ }\NormalTok{sumKeysImport }\OperatorTok\StringTok{ }\KeywordTok{filter}\NormalTok{(}\OperatorTok{!}\KeywordTok{grepl}\NormalTok{(}\StringTok{"Y"}\NormalTok{,is_sum_key))}
\end{Highlighting}
\end{Shaded}

To check, that the keys after the filter are all correct, we created a
small algorithm that checks for sum keys based on the the rules
described in the ``Sum Keys'' chapter. The code takes a while to run,
that is why we saved the result in a vector.

\begin{Shaded}
\begin{Highlighting}[]
\CommentTok{#filteredKeys <- c()}
\CommentTok{#for (key1 in captureKeys$key) \{}
\CommentTok{#  if (!key1 %in% filteredKeys) \{}
\CommentTok{#    for (key2 in captureKeys$key) \{}
\CommentTok{#      if (!key2 %in% filteredKeys) \{}
\CommentTok{#        if (key1 != key2) \{}
\CommentTok{#          if (regexpr(paste("^", str_remove(key2, "0+$"), sep = ""), key1)[1] == 1) \{}
\CommentTok{#            filteredKeys <- append(filteredKeys, key2)}
\CommentTok{#          \}}
\CommentTok{#        \}}
\CommentTok{#      \}}
\CommentTok{#    \}}
\CommentTok{#  \}}
\CommentTok{#\}}
\NormalTok{filteredKeys <-}\StringTok{ }\KeywordTok{c}\NormalTok{(}\StringTok{"133000"}\NormalTok{,}\StringTok{"141100"}\NormalTok{,}\StringTok{"231200"}\NormalTok{,}\StringTok{"305000"}\NormalTok{,}\StringTok{"310000"}\NormalTok{,}\StringTok{"315000"}\NormalTok{,}\StringTok{"325000"}\NormalTok{,}\StringTok{"326000"}\NormalTok{,}\StringTok{"335000"}\NormalTok{,}\StringTok{"340000"}\NormalTok{,}\StringTok{"345000"}\NormalTok{,}\StringTok{"350000"}\NormalTok{,}\StringTok{"390000"}\NormalTok{,}\StringTok{"435000"}\NormalTok{,}\StringTok{"436000"}\NormalTok{,}\StringTok{"620010"}\NormalTok{,}\StringTok{"655010"}\NormalTok{,}\StringTok{"670010"}\NormalTok{,}\StringTok{"670020"}\NormalTok{,}\StringTok{"670030"}\NormalTok{)}
\end{Highlighting}
\end{Shaded}

After checking the resulting keys manually we found, that the keys
``141100'', ``231200'' and ``133000'' were marked as capture keys, but
are sum keys, so these keys have to be removed from the capture keys and
added to the sum keys.

\begin{Shaded}
\begin{Highlighting}[]
\NormalTok{captureKeys <-}\StringTok{ }\NormalTok{sumKeysImport }\OperatorTok\StringTok{ }\KeywordTok{filter}\NormalTok{(}\KeywordTok{grepl}\NormalTok{(}\StringTok{"Y"}\NormalTok{,is_sum_key) }\OperatorTok{&}\StringTok{ }\OperatorTok{!}\NormalTok{(key }\OperatorTok\StringTok{ }\KeywordTok{c}\NormalTok{(}\StringTok{"141100"}\NormalTok{,}\StringTok{"231200"}\NormalTok{,}\StringTok{"133000"}\NormalTok{)))}
\NormalTok{sumKeys <-}\StringTok{ }\NormalTok{sumKeysImport }\OperatorTok\StringTok{ }\KeywordTok{filter}\NormalTok{(}\OperatorTok{!}\KeywordTok{grepl}\NormalTok{(}\StringTok{"Y"}\NormalTok{,is_sum_key) }\OperatorTok{|}\StringTok{ }\NormalTok{(key }\OperatorTok\StringTok{ }\KeywordTok{c}\NormalTok{(}\StringTok{"141100"}\NormalTok{,}\StringTok{"231200"}\NormalTok{,}\StringTok{"133000"}\NormalTok{)))}
\end{Highlighting}
\end{Shaded}

\hypertarget{filter-final-data-frames}{%
\subsubsection{Filter final data
frames}\label{filter-final-data-frames}}

First to reduce duplicate data, we filtered out all crime descriptions
into its own data frame, to it can be looked up afterwards.

\begin{Shaded}
\begin{Highlighting}[]
\NormalTok{crimeOverview <-}\StringTok{ }\NormalTok{originalSet }\OperatorTok\StringTok{ }\KeywordTok{select}\NormalTok{(key, crime) }\OperatorTok\StringTok{ }\KeywordTok{unique}\NormalTok{()}
\NormalTok{crimeOverview <-}\StringTok{ }\KeywordTok{arrange}\NormalTok{(crimeOverview, key)}
\KeywordTok{head}\NormalTok{(crimeOverview)}
\end{Highlighting}
\end{Shaded}

\begin{verbatim}
##      key
## 1 ------
## 2 ****00
## 3 ***100
## 4 ***200
## 5 ***300
## 6 ***400
##                                                                               crime
## 1                                                              Straftaten insgesamt
## 2                                                     Diebstahl insgesamt und zwar:
## 3             Diebstahl insgesamt von Kraftwagen einschl. unbefugte Ingebrauchnahme
## 4 Diebstahl insgesamt von Mopeds und Krafträdern einschl. unbefugte Ingebrauchnahme
## 5             Diebstahl insgesamt von Fahrrädern einschl. unbefugte Ingebrauchnahme
## 6                                              Diebstahl insgesamt von Schusswaffen
\end{verbatim}

\begin{Shaded}
\begin{Highlighting}[]
\NormalTok{originalSet <-}\StringTok{ }\NormalTok{originalSet }\OperatorTok\StringTok{ }\KeywordTok{select}\NormalTok{(}\OperatorTok{!}\NormalTok{crime)}
\end{Highlighting}
\end{Shaded}

\hypertarget{transform}{%
\subsection{Transform}\label{transform}}

In the final step before visualizing the data we need to check all the
types of the values in each column an change them to a type that is
usable for our analysis. We can also remove and create new Columns if
necessary. In the end we are also splitting the original data frame up
based on the provinces and the sum keys determined before.

\hypertarget{transforming-data-types}{%
\subsubsection{Transforming data types}\label{transforming-data-types}}

In this step we first check for the different data types we have in
every column. We expect to have characters as the key, a factor for the
provinces and every other column should be a number.

\begin{Shaded}
\begin{Highlighting}[]
\KeywordTok{sapply}\NormalTok{(originalSet, class)}
\end{Highlighting}
\end{Shaded}

\begin{verbatim}
##                   province                        key 
##                "character"                "character" 
##             detected_cases     detected_cases_percent 
##                  "numeric"                  "numeric" 
##         not_executed_cases not_executed_cases_percent 
##                  "numeric"                  "numeric" 
##     distribution_under_20K   distribution_20K_to_100K 
##                  "numeric"                  "numeric" 
##  distribution_100K_to_500K     distribution_over_500K 
##                  "numeric"                  "numeric" 
##       distribution_unknown             gun_threatened 
##                  "numeric"                  "numeric" 
##                   gun_shot               solved_cases 
##                  "numeric"                  "numeric" 
##       solved_cases_percent                   suspects 
##                  "numeric"                  "numeric" 
##                       male                     female 
##                  "numeric"                  "numeric" 
##                 non_german         non_german_percent 
##                  "numeric"                  "numeric" 
##                inhabitants 
##                  "numeric"
\end{verbatim}

Except for the province all values have the type expected, the next step
is to transform the province into a factor.

\begin{Shaded}
\begin{Highlighting}[]
\NormalTok{originalSet}\OperatorTok{$}\NormalTok{province <-}\StringTok{ }\KeywordTok{as.factor}\NormalTok{(originalSet}\OperatorTok{$}\NormalTok{province) }
\KeywordTok{sapply}\NormalTok{(originalSet, class)}
\end{Highlighting}
\end{Shaded}

\begin{verbatim}
##                   province                        key 
##                   "factor"                "character" 
##             detected_cases     detected_cases_percent 
##                  "numeric"                  "numeric" 
##         not_executed_cases not_executed_cases_percent 
##                  "numeric"                  "numeric" 
##     distribution_under_20K   distribution_20K_to_100K 
##                  "numeric"                  "numeric" 
##  distribution_100K_to_500K     distribution_over_500K 
##                  "numeric"                  "numeric" 
##       distribution_unknown             gun_threatened 
##                  "numeric"                  "numeric" 
##                   gun_shot               solved_cases 
##                  "numeric"                  "numeric" 
##       solved_cases_percent                   suspects 
##                  "numeric"                  "numeric" 
##                       male                     female 
##                  "numeric"                  "numeric" 
##                 non_german         non_german_percent 
##                  "numeric"                  "numeric" 
##                inhabitants 
##                  "numeric"
\end{verbatim}

\hypertarget{remove-calculated-columns}{%
\subsubsection{Remove calculated
columns}\label{remove-calculated-columns}}

Some columns in the data set are just calculated from the given data, so
to remove the possibility of a calculation error, we remove these
columns and add them on our own, once they are needed in the analysis.

\begin{Shaded}
\begin{Highlighting}[]
\NormalTok{originalSet <-}\StringTok{ }\NormalTok{originalSet }\OperatorTok\StringTok{ }\KeywordTok{select}\NormalTok{(key, province, inhabitants, detected_cases, solved_cases, not_executed_cases, suspects, male, female, non_german, gun_threatened, gun_shot, distribution_under_20K, distribution_20K_to_100K, distribution_100K_to_500K, distribution_over_500K, distribution_unknown)}
\end{Highlighting}
\end{Shaded}

\hypertarget{split-up-data}{%
\subsubsection{Split up data}\label{split-up-data}}

This could also be part of the tidy step, but to reduce the amount of
duplicate code for all checks, we split the data at the end. Therefore,
we merge both key frames together with the original data set to separate
the sum keys and the capture keys into separate data frames.

\begin{Shaded}
\begin{Highlighting}[]
\NormalTok{captureValues <-}\StringTok{ }\KeywordTok{merge}\NormalTok{(originalSet, captureKeys) }\OperatorTok\StringTok{ }\KeywordTok{select}\NormalTok{(}\OperatorTok{!}\NormalTok{is_sum_key)}
\NormalTok{sumValues <-}\StringTok{ }\KeywordTok{merge}\NormalTok{(originalSet, sumKeys) }\OperatorTok\StringTok{ }\KeywordTok{select}\NormalTok{(}\OperatorTok{!}\NormalTok{is_sum_key)}
\end{Highlighting}
\end{Shaded}

With the sum keys now separated, we also divide both data frames into
the values for the provinces and the values for whole Germany.

\begin{Shaded}
\begin{Highlighting}[]
\NormalTok{germanCaptureSet <-}
\StringTok{  }\NormalTok{captureValues }\OperatorTok\StringTok{ }\KeywordTok{filter}\NormalTok{(}\KeywordTok{grepl}\NormalTok{(}\StringTok{"Bundesrepublik Deutschland"}\NormalTok{, province))}
\NormalTok{provinceCaptureSet <-}
\StringTok{  }\NormalTok{captureValues }\OperatorTok\StringTok{ }\KeywordTok{filter}\NormalTok{(}\OperatorTok{!}\KeywordTok{grepl}\NormalTok{(}\StringTok{"Bundesrepublik Deutschland"}\NormalTok{, province))}

\NormalTok{germanSumSet <-}
\StringTok{  }\NormalTok{sumValues }\OperatorTok\StringTok{ }\KeywordTok{filter}\NormalTok{(}\KeywordTok{grepl}\NormalTok{(}\StringTok{"Bundesrepublik Deutschland"}\NormalTok{, province))}
\NormalTok{provinceSumSet <-}
\StringTok{  }\NormalTok{sumValues }\OperatorTok\StringTok{ }\KeywordTok{filter}\NormalTok{(}\OperatorTok{!}\KeywordTok{grepl}\NormalTok{(}\StringTok{"Bundesrepublik Deutschland"}\NormalTok{, province))}
\end{Highlighting}
\end{Shaded}

To check that the sum keys from before where separated correct we check
that the sum of all detected cases in each German set is equal to the
sum for the detected cases in the province set.

\begin{Shaded}
\begin{Highlighting}[]
\KeywordTok{sum}\NormalTok{(germanCaptureSet}\OperatorTok{$}\NormalTok{detected_cases) }\OperatorTok{==}\StringTok{ }\KeywordTok{sum}\NormalTok{(provinceCaptureSet}\OperatorTok{$}\NormalTok{detected_cases)}
\end{Highlighting}
\end{Shaded}

\begin{verbatim}
## [1] TRUE
\end{verbatim}

\begin{Shaded}
\begin{Highlighting}[]
\KeywordTok{sum}\NormalTok{(germanSumSet}\OperatorTok{$}\NormalTok{detected_cases) }\OperatorTok{==}\StringTok{ }\KeywordTok{sum}\NormalTok{(provinceSumSet}\OperatorTok{$}\NormalTok{detected_cases)}
\end{Highlighting}
\end{Shaded}

\begin{verbatim}
## [1] TRUE
\end{verbatim}

\hypertarget{visualise}{%
\subsection{Visualise}\label{visualise}}

\begin{Shaded}
\begin{Highlighting}[]
\KeywordTok{options}\NormalTok{(}\DataTypeTok{scipen=}\DecValTok{999}\NormalTok{)}
\end{Highlighting}
\end{Shaded}

\begin{Shaded}
\begin{Highlighting}[]
\KeywordTok{plot}\NormalTok{(germanCaptureSet)}
\end{Highlighting}
\end{Shaded}

\includegraphics{firstTry_files/figure-latex/unnamed-chunk-2-1.pdf}

\begin{Shaded}
\begin{Highlighting}[]
\KeywordTok{filter}\NormalTok{(crimeOverview, key }\OperatorTok{==}\StringTok{ "731800"}\NormalTok{)}
\end{Highlighting}
\end{Shaded}

\begin{verbatim}
##      key                                              crime
## 1 731800 Allgemeiner Verstoß mit Cannabis und Zubereitungen
\end{verbatim}

\begin{Shaded}
\begin{Highlighting}[]
\KeywordTok{ggplot}\NormalTok{(}
  \DataTypeTok{data =} \KeywordTok{filter}\NormalTok{(germanCaptureSet, detected_cases}\OperatorTok{/}\NormalTok{inhabitants }\OperatorTok{>}\StringTok{ }\FloatTok{0.001}\NormalTok{),}
  \DataTypeTok{mapping =} \KeywordTok{aes}\NormalTok{(detected_cases}\OperatorTok{/}\NormalTok{inhabitants, solved_cases}\OperatorTok{/}\NormalTok{detected_cases)) }\OperatorTok{+}
\StringTok{  }\CommentTok{#geom_point() +}
\StringTok{  }\CommentTok{#geom_smooth(method="lm")}
\StringTok{  }\KeywordTok{geom_text}\NormalTok{(}\DataTypeTok{mapping =} \KeywordTok{aes}\NormalTok{(}\DataTypeTok{label =}\NormalTok{ key))}
\end{Highlighting}
\end{Shaded}

\includegraphics{firstTry_files/figure-latex/unnamed-chunk-4-1.pdf}

\hypertarget{model}{%
\subsection{Model}\label{model}}

\end{document}
